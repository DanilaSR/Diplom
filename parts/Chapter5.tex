\chapter{Заключение}
\label{sec:Chapter5} \index{Chapter5}

	В данной работе был предложен метод определения качества синтеза РЛИ, основанный на анализе отклика от псевдоточечных целей. Представленная модель формирования отклика от псевдоточечных целей показала равенство между максимумом распределения разрешений псевдоточечных целей и разрешением точечной цели. Был разработан алгоритм поиска псевдоточечных целей на РЛИ, основанный на вычислении АКФ и коэффициента эксцесса. Также были разработаны алгоритмы анализа отклика от псевдоточечной цели и статистической оценке распределения разрешений. Разработанные алгоритмы были реализованы на языках Matlab и Python. Валидация разработанных алгоритмов была проведена на полигоне с уголковыми отражателями Surat (Австралия). Итоговая погрешность в определении разрешений по дальности и по азимуту составила 0.8\% и 2.37\% соответсвенно по дальности и по азимуту. Валидация на снимке Sentinel показала возможность работы со снимками без апприорной информации о снимаемой сцене. Исходя из постановки задачи исследования все требования были выполнены. 

	В дальнейшем планируется увеличить точность метода с помощью введения показателя точечности псевдоточечных целей и написание промышленного продукта для регулярного использования как в научных целей, так и в практических приложениях.

\newpage
