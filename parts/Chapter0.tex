\chapter{Введение}
\label{sec:Chapter0} \index{Chapter0}

\section{Вступление}

	В настоящее время радиолокаторы с синтезированной апертурой (РСА) используются для решения различных задач по мониторингу поверхности Земли. В отличие от оптической съемки, РСА обеспечивает всепогодное круглосуточное наблюдение за земной поверхностью [1]. В процессе работы радиолокатора выполняется периодическое излучение зондирующего импульса. В интервалах между стробами излучения выполняется запись отраженного от поверхности Земли сигнала, на основе которого формируется радиоголограмма [1]. При помощи алгоритмов синтеза из записанной радиоголограммы формируется радиолокационное изображение (РЛИ).

	Потребность в оперативной оценке качества РЛИ возникает как у конечных потребителей, так и у разработчиков. В виду наличия разных дестабилизирующих факторов – неоднозначности сигнала, нелинейности тракта, аппаратурных нестабильностей, погрешностей траекторных отклонений, качество РЛИ сильно ухудшается. При наличии неисправностей в сквозном тракте можно выполнить корреляционный анализ и на этапе постобработки программно их исправить.	
	
	Согласно мировой практике для оценки качества РЛИ проводят съемку полигонов, оборудованных измерительными мирами из уголковых отражателей или используют транспондеры. Для космических РСА периодичность съемки, зависящая от параметров орбиты и реализуемой полосы обзора, обычно составляет несколько суток. Поэтому для достижения оперативности в калибровке и верификации снимков необходимо иметь средства оценки качества РЛИ непосредственно по материалам текущей съемки.	
	
	Актуальность данной работы заключается в следующем:
	
	1. Необходимость подтверждения качества РЛИ в отсутствие априорной информации о снимаемой сцене и отсутствие калибровочных целей (уголковых отражателей, транспондеров). В особенности РЛИ, полученные с самолетных РСА требуют постоянного контроля вследствие большого влияния траекторных искажений.
	
	2. Также данная работа может использоваться для оценки качества работы алгоритмов синтеза в различных исследовательских задачах.
	
	Таким образом, целью данной работы является разработка алгоритмов для оценки качества синтеза РЛИ по материалам текущей съемки в отсутствие априорной информации о снимаемой сцене.

\section{Термины и определения}
	В настоящем отчете о НИР применяют следующие термины с соответствующими определениями:\\
	\textbf{Радиолокационное зондирование} - наблюдение объектов в радиодиапазоне волн с детальностью оптических систем. В отличие от оптических систем системы радиовидения дают возможность получать изображения объектов независимо от метеоусловий и естественной освещенности, на значительном удалении и одновременно в широкой зоне обзора, в том числе объектов, невидимых в оптическом дмапозоне волн\\	
	\textbf{Синтезированная апертура} - метод получения высокого разрешения по углу при малых размерах антенны путём запоминания отраженного от объекта электромагнитного поля\\
	\textbf{Точечный отражатель} - одиночный, изолированный рассеиватель\\
	\textbf{Псевдоточечная цель} - объекты, которые не являются одиночными, изолированным рассеивателями, но при этом сохраняют диффракционный отклик \\
\section{Перечень сокращений и обозначений}
	В настоящем отчете о НИР применяют следующие сокращения и обозначения:\\	
	\textbf{РЛИ} - радиолокационное изображение\\
	\textbf{РСА} - радиолокатор с синтезированной апертурой\\
	\textbf{РЛС} - радиолокационные системы\\
	\textbf{ФРТ} - функция рассеяния точки\\
	\textbf{АКФ} - автокорреляционная функция\\


\newpage
